\begin{frame}
  \frametitle{\textbf{$b$-jet Indetification}}
  \begin{columns}
    \column{0.5\textwidth}
    \begin{itemize}
    \item Many reconstructed objects can be used to discriminate between $b$ and light-jets
      \begin{itemize}
      \item Tracks
      \item Vertices
      \item Leptons
      \end{itemize}
    \item 3D impact paramater (IP)
      \begin{itemize}
      \item Track's distance-of-closest-approach to jet-axis
      \item Positive values: particle travelling along jet-axis
      \item $b$-quarks tend to have large IP because of longer lifetime
      \end{itemize}
    \item Combined-secondary-vertex (CSV) discriminator
      \begin{itemize}
      \item Combines vertex (and secondary vertex) information with track-based lifetime info in a complicated algorithm
      \item $b$-jets tend to contain secondary vertices
      \end{itemize}
    \item \textbf{Muon rel-}$\boldmath{p_{\text{T}}}$
      \begin{itemize}
      \item Measures muon's transverse momentum relative to jet-axis
      \item $b$-quarks tend to impart more transverse momentum into daughter muons
      \end{itemize}
    \end{itemize}
    \column{0.5\textwidth}
    \centering
    \begin{tikzpicture}
      \node{\includegraphics[width=0.58\textwidth]{b-IP.png}};
    \end{tikzpicture}
    \begin{tikzpicture}
      \node{\includegraphics[width=0.58\textwidth]{csv-discriminator.png}};
    \end{tikzpicture}
    \begin{tikzpicture}
      \node{\includegraphics[width=0.58\textwidth]{muon-rel-pt.png}};
    \end{tikzpicture}
  \end{columns}
\end{frame}
